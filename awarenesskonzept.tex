\documentclass{article}

% Language setting
% Replace `english' with e.g. `spanish' to change the document language
\usepackage[german]{babel}

% Set page size and margins
% Replace `letterpaper' with `a4paper' for UK/EU standard size
\usepackage[letterpaper,top=2cm,bottom=2cm,left=3cm,right=3cm,marginparwidth=1.75cm]{geometry}

% Useful packages
\usepackage{amsmath}
\usepackage{graphicx}
\usepackage[colorlinks=true, allcolors=blue]{hyperref}
\usepackage{parskip} % kein hängender Einzug

\newcommand{\comment}[1]{}

\title{Awareness-Konzept}
\author{Fachschaften der INPhiMa}

\begin{document}
\maketitle

\section{Code of Conduct}

\subsection{Ziele}
Die Fachschaften der INPhiMa möchten einen Raum schaffen, in dem sich alle Menschen willkommen fühlen.

Das bedeutet insbesondere einen respektvollen und diskriminierungsfreien Umgang miteinander, unabhängig von unter anderem Geschlecht, Religion, Aussehen, sexueller Orientierung, Behinderung, Sprache, sozialem oder wirtschaftlichem Status. Dazu beitragen soll dieser Code of Conduct (Verhaltenskodex). Er gilt für alle physischen und digitalen Räume der Fachschaft, unter anderem die Fachschaftsräume und -veranstaltungen, die Mailinglisten und die Chaträume.

\subsection{Umgang miteinander}
Die Fachschaften der INPhiMa erwarten auf all ihren Veranstaltungen:
\begin{itemize}
    \item einen respektvollen und höflichen Umgang miteinander
    \item eine direkte, offene und konstruktive Kommunikation miteinander
    \item eine diskriminierungsfreie Ausdrucksweise
    \item das Respektieren und Nutzen der gewünschten Anrede und Pronomen 
    \item das Respektieren von Grenzen Anderer (Nur Ja heißt Ja)
\end{itemize}
Es wird erwartet, dass bei problematischen Verhaltensweisen eingeschritten wird. Dies ermöglicht das Lösen von Konflikten im besten Fall früh und einfach.

\subsection{Was nicht toleriert wird}
Die Fachschaften der INPhiMa tolerieren kein grenzüberschreitendes, diskriminierendes oder verletzendes Verhalten. Dazu zählen unter anderem:
\begin{itemize}
    \item verletzende Bemerkungen
    \item unerwünschtes Fotografieren und Aufnehmen
    \item aufdrängen von Alkohol und andere Substanzen
    \item vorsätzliches Verbreiten eines Teils der Identität einer Person ohne deren Einverständnis (Outen)
    \item absichtliches Misgendern und das Nennen von Dead Names
    \item veröffentlichen von anstößigen Bildern
    \item anstößiges Verhalten
    \item physische/seelische Gewalt, Androhung und Anstiftung zu dieser
    \item Mobbing
    \item unerwünschte Bemerkungen bezüglich der Lebensweise einer Person
    \item physische Berührungen oder das Andeuten von Berührungen ohne vorherige Zustimmung
    \item übermäßiger Alkoholkonsum oder anderer Drogenkonsum
\end{itemize}

Außerdem gilt die Definitionsmacht der Betroffenen: Jede Person hat individuelle Grenzen, sodass die Wahrnehmung und Erfahrung ganz unterschiedlich sein kann. Was eine betroffene Person als Grenzüberschreitung und/oder Diskriminierung wahrgenommen hat, wird als solche behandelt, auch wenn es nicht der eigenen Wahrnehmung entspricht.

\subsection{Geltungsbereich}
Dieser Code of Conduct ist für alle Bereiche der Fachschaften der INPhiMa gültig, online wie offline.

Wir werden alle ehrlichen Meldungen über Belästigungen durch Mitglieder der Fachschaft – speziell auch Belästigungen durch Teile des Fachschaftsrats und Ratsmitglieder – ernst nehmen. Dies schließt Belästigung außerhalb unserer Räume und zu einem beliebigen Zeitpunkt mit ein.

\subsection{Konsequenzen}
Studierende und Gäste, die gebeten werden, belästigendes Verhalten zu unterlassen, haben dieser Aufforderung sofort nachzukommen.

Alle Teilnehmenden auf Veranstaltungen der Fachschaften der INPhiMa sind dafür verantwortlich, auf die Einhaltung dieses Code of Conducts zu achten. Als unmittelbare Maßnahme haben Ratsmitglieder die Möglichkeit, vom Hausrecht Gebrauch zu machen und Personen kurzzeitig unserer Räume und Veranstaltungen zu verweisen. Hierzu kann zur Durchsetzung ggf. die Security oder die Polizei dazugezogen werden. Die gilt vor allem bei Verstößen, denen eine Straftat vorhergegangen ist.
 
Des Weiteren können die Fachschaftsräte der INPhiMa bei gröberen Verstößen oder anhaltenden Verstö\-ßen ein Veranstaltungsverbot aussprechen. Dieses Verbot kann von einem Fachschaftsrat ausgesprochen werden und gilt für alle anderen Fachschaften. 

Bei Alkoholmissbrauch wird ggf. der Notdienst kontaktiert, falls dies von Ratsmitglieder der INPhiMa als notwendig angesehen wird. Bei wiederholtem Alkoholmissbrauch auf Veranstaltungen von den Fachschaften der INPhiMa wird ein Alkoholverbot ausgesprochen. Dies kann in Einzelfällen aufgehoben werden. Bei Nicht-Einhalten des Alkoholverbots wird ein Veranstaltungsverbot ausgesprochen.

\section{Weiteres}

\subsection{Awareness auf INPhiMa-Veranstaltungen}
Awareness-Personen auf INPhiMa Veranstaltungen tragen Blumenketten um besser von Studierenden erkannt zu werden. Jedes Ratsmitglied gilt als Ansprechperson bei Problemen, doch die ausgewählten Personen sind geschult und/oder werden von der INPhiMa als vertrauenswürdig eingeschätzt. Aktive Awareness-Personen sind angehalten auf Veranstaltungen kein bis wenig Alkohol (oder anderweitig das Urteilsvermögen einschränkende Substanzen) zu konsumieren.

Auf jeder INPhiMa-Veranstaltung sind, sofern möglich, immer 2 Awareness-Personen, die nach Mög\-lich\-keit unterschiedlicher Geschlechter sind, wovon mindestens eine Person FLINTA*-Person ist. %primär weiblich gelesen, aber ist schwer eine diskriminierungsfreie Wortwahl zu finden.

Bei Veranstaltungen mit mehreren Veranstaltungsorten (z.B. Campusrallye, Kneipentouren, etc.) müs\-sen die Awareness-Personen nicht dauerhaft vor Ort sein. Im Vorfeld wird hier eine Erreichbarkeit und, bei Möglichkeit, eine schnelle Verfügbarkeit vor Ort verabredet.  

Auf Veranstaltungen, auf denen Alkohol angeboten wird, ist der Einsatz eines Awarenessteams verpflichtend.

\subsection{Awarenessfälle mit involvierten Ratsmitgliedern}
Bei schwerwiegenden Awarenessfällen, und insbesondere bei Fällen zwischen Ratsmitgliedern und Fachschaftsräten der INPhiMa, zieht das Awareness-Komitee zur Klärung das FS-Ref mit ein. Das FS-Ref ist in dem Fall dann mit dem Awareness-Komitee gleichgestellt (zusätzlich zu ihrer erhöhten Stellung als FS-Ref).

\subsection{Aufgaben vom Awareness-Komitee}
Alle Fachschaften der INPhiMa haben ausgewählte Personen für das Awareness-Komitee. Diese Personen dienen als universelle Ansprechpartner*innen auch außerhalb von Veranstaltungen.

Das Awareness-Komitee kommt bei Bedarf und mindestens einmal im Semester zusammen und bespricht anonymisiert das Vorgehen bei Fällen, bei denen es Klärungsbedarf gibt. Außerdem berät es über Konsequenzen und allgemeine awareness-relevante Themen aus der INPhiMa und gibt handlungsweisende Empfehlungen in die INPhiMa zurück.  

Konkrete Personenbezogenen Infos zu Betroffenen werden auf das Nötigste begrenzt und verlassen, ohne Zustimmung der betroffenen Person, niemals das Komitee. Jegliche Awareness-Komitee-Protokolle sind nur den gewählten Awareness-Personen zugänglich zu machen.

Ausschlüsse von Veranstaltungen werden durch die Awarenesspersonen und dem Awareness-Komitee nachbereitet und der Kontakt zu den betroffenen Personen (sowohl Opfer, als auch der diskriminiereden/gewaltausübende Person (dgP)) gesucht, um die Situation nachzubereiten. Sollte dies von den betroffenen Personen nicht gewünscht sein, dann wird dies respektiert aber auch für einen späteren Zeitpunkt weiterhin angeboten.

Nach der Neuwahl des Komitees findet eine geordnete Übergabe zusammen mit dem alten und dem neuen Awareness-Komitee statt, bei der unter anderem aktuell geltende Ausschlüsse und andere wichtige Informationen ausgetauscht werden.

\subsection{Kontakt zu Awareness-Personen}
Die Awareness-Personen können außerhalb von Veranstaltungen über die öffentlich einsehbaren Kontaktmöglichkeiten kontaktiert werden. % Sichere Messenger (Signal, Threema oder Matrix) sollten von mindestens einer Awareness-Person aus der Fachschaft abgedeckt werden.
Außerdem gibt es die Möglichkeit über ein Formular anonym das Awareness-Komitee zu kontaktieren. Nur die Personen aus dem Komitee haben Zugriff auf das Formular. % Evtl. Cryptpad?

Die Kontaktmöglichkeiten sind sichtbar in den FS-Räumen, sowie auf Veranstaltungen auszulegen. Vordrucke hierzu finden sich in der INPhiMa-Cloud. 

\subsection{Umgang mit Meldungen}
\subsubsection{Meldungen außerhalb von Veranstaltungen}
Alle Ratsmitglieder sind grundsätzlich für betroffene Personen ansprechbar. Diese kontaktieren, auf Wunsch der betroffenen Person, umgehend eine Person aus dem Awareness-Komitee. 

Außerdem sind alle Awareness-Personen auf den jeweiligen Websites zu finden. Dort gibt es sowohl eine direkte Kontaktmöglichkeit, als auch ein anonymes Formular, auf welches nur die Awareness-Personen der jeweiligen Fachschaft Zugriff haben. Auf der INPhiMa-Seite finden sich alle Mitglieder des Awareness-Komitees, geordnet nach Fachschaft, sowie ein anonymes Formular, auf welches nur das Awareness-Komitee Zugriff hat.

\subsubsection{Umgang mit der um Hilfe bittenden Person}

Die Awareness-Person:
\begin{itemize}
    \item hört der betroffenen Person zu und nimmt sie ernst.
    \item klärt, welche Art von Unterstützung sich die Person wünscht.
    \item bietet einen Rückzugsraum an.
    \item bietet an, dass sich die betroffene Person nicht selbst mit der beschuldigten Person auseinandersetzen muss.
    \item bleibt ansprechbar für die Person und signalisiert dies.
\end{itemize}
Unser Verhalten:
\begin{itemize}
    \item Wir sind vorsichtig mit Körperkontakt, außer er wird ausdrücklich erwünscht.
    \item Wir sind vorsichtig mit Fragen.
    \item Wir geben der betroffenen Person Raum und Zeit.
    \item Wir stellen unsere eigenen Wünsche und Bedürfnisse hinten an und respektieren die Entscheidungen der betroffenen Person.
    \item Wir respektieren es, wenn die betroffene Person keine Unterstützung möchte, aber bieten dennoch an, dass wir im Laufe der Veranstaltung/Tage noch ansprechbar sind.
\end{itemize}

\subsubsection{Transformative Justice}

In der INPhiMa streben wir das Prinzip der transformativen Täter*innenarbeit an.
Dies bedeutet, dass wir die gewaltausübende/diskriminiereden Person (gdP) dazu ermutigen, 
dass sie sich mit ihrem Verhalten aktiv auseinandersetzen und sich reflektieren,
mit dem Ziel, dass vollständige Ausschlüsse seltener ausgesprochen werden und die Person wieder in die Gruppe integriert zu werden.\\
Vollständige Ausschlüsse und schwerere Konsequenzen sind jedoch weiterhin möglich.\\

Hierfür arbeitet das Awareness-Komitee zusammen mit der um Hilfe bittenden Person
und der gdP ein Konzept aus, wie die Person sich
reflektieren kann und welche Bedingungen es gibt, um wieder in die Gruppe integriert zu werden.\\

Sollte sich die gdP sich weigern an diesem Prozess teilzunehmen oder sich nicht an die Bedingungen halten,
kann das Awareness-Komitee einen längerfristigen Ausschluss vorschlagen.\\
Dies wirkt sich nicht darauf aus, dass das Awareness-Komitee der Person weiterhin Unterstützung anbietet.\\

\subsection{Grenzen der Awareness-Personen}

\subsubsection{Kein Ausschlussgremium}
Die Awareness-Personen und das Awareness-Komitee haben, wie alle INPhiMa-Ratsmitglieder, die Möglichkeit vom Hausrecht gebrauch zu machen, aber sind nicht bemächtigt eigenständig Personen von allen INPhiMa-Veranstaltungen auszuschließen. Hierfür können sie aber die INPhiMa beraten.

\subsubsection{Wir sind keine Polizei}
Die Awareness-Personen und das Komitee sind kein Ersatz für die Einschaltung der Polizei. In besonders schweren Fällen oder auf Wunsch der Betroffenen, ist das Komitee und/oder die Awareness-Person angehalten die Polizei dazuzuholen, auch um die Sicherheit anderer Teilnehmden zu gewährleisten.

\subsubsection{Keine Psychologen}
Die Awareness-Personen sind keine ausgebildeten Psychologen. Sie können und sollen keine professionelle Psychotherapie ersetzen und ist nur in einem sehr begrenztem Rahmen fähig bei schweren psychischen Problemen zu helfen. Bei Fällen, die die Handlungskompetenzen übersteigen ist die Betroffene Person auf professionelle Hilfe (Selbsthilfehotline, Krisentelefon, Notruf, 116117) zu verweisen. 

\subsubsection{Keine professionelle Mediation}
Die Awareness-Personen stellen keinen Ersatz von professioneller Mediation bei schwerwiegenden Streitigkeiten und Konflikten dar. 

\subsubsection{Wir sind parteiisch}
Die Awareness-Personen ergreifen immer Partei für die betroffenden Personen. Die Wünsche und Bedürfnisse der betroffenen Person definieren die Handlungsweisen und Arbeit. Im Komitee und Plenum der Fachschaften und INPhiMa stehen die Awareness-Personen für die betroffene Person ein. Täter*innenschutz\footnote{Täter*innenschutz bezeichnet hier die Deckung, Verharmlosung oder Tolerierung des Verhaltens von dgPs.} wird nicht toleriert. Die Unschuldsvermutung gilt auch für Betroffene.

\subsubsection{Wir kennen unsere Grenzen}
Awareness-Personen kommunizieren, sowohl auf Veranstaltungen, als auch außerhalb dieser, offen, wenn sie sich gerade nicht Wohl mit einem Thema oder Fall fühlen. Sie holen sich dann eine weitere Person zur Hilfe dazu oder geben den Fall komplett an eine andere Awareness-Person ab. 

\end{document}