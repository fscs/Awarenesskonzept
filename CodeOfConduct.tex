\documentclass{article}
\usepackage{graphicx} % Required for inserting images

\title{Code of Conduct}
\begin{document}

\maketitle

\section{Unser Ziel}
Wir sehen es als unsere Aufgabe, eine freundliche, sichere und einladende Umgebung zu schaffen, unabhängig vom Geschlecht, sexueller Orientierung, Befähigung, Herkunft, Religion (oder deren Absenz), sowie gesellschaftlichem und wirtschaftlichen Status.
\\
Dieser Code of Conduct fasst unsere Erwartungen an allen Beteiligten unserer Veranstaltungen oder unseres Umfeldes zusammen. Wir laden dazu ein, sichere und positive Erfahrungen für alle mitzugestalten.

\section{Erwartetes Verhalten}
\begin{itemize}
    \item Sei rücksichtsvoll und respektvoll in Wort und Tat.
    \item Bemühe dich um Zusammenarbeit, damit du Konflikte von Anfang an vermeiden kannst.
    \item Nimm Abstand von erniedrigender, diskriminierender oder belästigender Sprache und Verhalten.
    \item Achte auf deine Umgebung und die anderen Teilnehmenden. Mache die Veranstaltenden oder andere Anwesende darauf aufmerksam, wenn du eine gefährliche Situation, jemanden in Bedrängnis oder Verletzungen dieses Verhaltenskodexes bemerkst, selbst wenn sie zunächst belanglos erscheinen.
\end{itemize}

\section{Inakzeptables Verhalten}
Inakzeptable Verhaltensweisen beinhalten: Einschüchterung, Belästigung, Mobbing, übermäßiger Drogen- und Alkoholkonsum sowie dessen Zwang dazu, beleidigende, diskriminierende, abwertende oder erniedrigende Sprache und Verhalten durch jegliche Teilnehmenden
\\ 
Dies gilt online, auf allen zugehörigen Veranstaltungen und in persönlichen Gesprächen im Rahmen der Fachschaften.
\\
Belästigung beinhaltet: Verletzende oder abwertende mündliche oder schriftliche Kommentare in Bezug auf Geschlecht, sexuelle Orientierung, Abstammung, Religion, Behinderung; unangemessene Verwendung von Nacktheit und/oder sexuellem Bildmaterial (inklusive Präsentationsslides); absichtliche Einschüchterung, Stalking oder Nachlaufen; belästigendes Fotografieren oder Filmen; ständige Unterbrechung von Vorträgen oder anderen Events; unangemessener Körperkontakt und unerwünschte sexuelle Zuwendung

\section{Folgen von inakzeptablem Verhalten}
Inakzeptables Verhalten jedlicher Teilnehmenden (externe Personen, Studierende, Ratsmitglieder) wird nicht toleriert.
\\
Wenn ein Person sich auf inakzeptable Art und Weise verhält, steht es den Veranstaltenden zu, jegliche ihnen angemessen erscheinende Maßnahme zu ergreifen, bis einschließlich einem befristeten oder permanenten Ausschluss aus Fachschaftsräume und -veranstaltung ohne Warnung.

\section{Wenn du inakzeptables Verhalten erlebst}
Wenn du von inakzeptablem Verhalten betroffen bist, dieses beobachtest oder andere Anliegen hast, teile dies bitte so bald wie möglich einer für die Veranstaltung verantwortlichen Person mit.
\\
\textbf{TODO Telefonnummer}

\section{Behandlung von Beschwerden}
Wenn du dich zu Unrecht oder auf ungerechte Art und Weise beschuldigt fühlst, diesen Code of Conduct verletzt zu haben, wende dich bitte mit einer genauen Beschreibung deiner Beschwerde an eine für die entsprechende Veranstaltung verantwortliche Person oder die Awareness-Personen. Deine Beschwerde wird dann in Übereinstimmung mit unseren vorhandenen Richtlinien behandelt.

\section{Geltungsbereich}
Wir erwarten, dass sich alle Personen an jedweden Veranstaltungsorten - online und offline - sowie in allen persönlichen Gesprächen im Rahmen der Fachschaften an diesen Verahltenskodex halten.

\section{Lizenz}
Der Berlin Code of Conduct steht unter der Creative Commons Attribution-ShareAlike 4.0 International (CC BY-SA 4.0) Lizenz. Er basiert auf dem pdx.rb Code of Conduct.
\end{document}